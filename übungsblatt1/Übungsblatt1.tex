\documentclass[10pt, a4paper]{article}
 \usepackage[german,ngerman]{babel}
 \usepackage[utf8]{inputenc}
 \usepackage[T1]{fontenc}
 \usepackage{listings}
\usepackage{color}

\definecolor{mygreen}{rgb}{0,0.6,0}
\definecolor{mygray}{rgb}{0.5,0.5,0.5}
\definecolor{mymauve}{rgb}{0.58,0,0.82}

\lstdefinestyle{customc}{
  belowcaptionskip=1\baselineskip,
  breaklines=true,
  frame=L,
  xleftmargin=\parindent,
  language=C,
  showstringspaces=false,
  basicstyle=\footnotesize\ttfamily,
  %keywordstyle=\bfseries\color{green!40!black},
  %commentstyle=\itshape\color{purple!40!black},
  identifierstyle=\color{blue},
  %stringstyle=\color{orange},
}

\lstset{ %
  backgroundcolor=\color{white},   % choose the background color; you must add \usepackage{color} or \usepackage{xcolor}; should come as last argument
  basicstyle=\footnotesize,        % the size of the fonts that are used for the code
  breakatwhitespace=false,         % sets if automatic breaks should only happen at whitespace
  breaklines=true,                 % sets automatic line breaking
  captionpos=b,                    % sets the caption-position to bottom
  commentstyle=\color{mygreen},    % comment style
  deletekeywords={...},            % if you want to delete keywords from the given language
  escapeinside={\%*}{*)},          % if you want to add LaTeX within your code
  extendedchars=true,              % lets you use non-ASCII characters; for 8-bits encodings only, does not work with UTF-8
  %frame=single,	                   % adds a frame around the code
  keepspaces=true,                 % keeps spaces in text, useful for keeping indentation of code (possibly needs columns=flexible)
  keywordstyle=\color{blue},       % keyword style
  %language=customc,                 % the language of the code
  morekeywords={*,...},            % if you want to add more keywords to the set
  %numbers=left,                    % where to put the line-numbers; possible values are (none, left, right)
  numbersep=5pt,                   % how far the line-numbers are from the code
  numberstyle=\tiny\color{mygray}, % the style that is used for the line-numbers
  rulecolor=\color{black},         % if not set, the frame-color may be changed on line-breaks within not-black text (e.g. comments (green here))
  showspaces=false,                % show spaces everywhere adding particular underscores; it overrides 'showstringspaces'
  showstringspaces=false,          % underline spaces within strings only
  showtabs=false,                  % show tabs within strings adding particular underscores
  stepnumber=2,                    % the step between two line-numbers. If it's 1, each line will be numbered
  stringstyle=\color{mymauve},     % string literal style
  tabsize=2,	                   % sets default tabsize to 2 spaces
  title=\lstname                   % show the filename of files included with \lstinputlisting; also try caption instead of title
}

\title{Informatik 1 Übungsblatt 1 Lösungen}
\author{Felix Weilbach}
\date{\today}

\begin{document}
\lstset{language=C}
\maketitle

 \section{Aufgabe 1} 
 \label{sec:aufgabe-1}
a) 
\begin {itemize}
  \item int a, j;
  \item char a = '<';
  \item 0,01256
  \item unsigned int
  \item ja, float
  \item nein, wegen -- muss heißen -1
  \item ja, char
  \item 2
  \item double x = pow(x, 2);
\end{itemize}
b)
\begin {itemize}
  \item Sorgt für Feldbreite drei bei Ausgabe.
  \item Sorgt für mindestens drei Stellen nach Komma bzw. Punkt.
  \item Nein, + ist Illegal.
  \item Ja.
  \item 1.0
  \item 001.1
\end{itemize}
c)
\begin {itemize}
  \item  ./a -x test
  \item gcc a.c -o Programm 
\end{itemize}

\section{Aufgabe 2} 
 \label{sec:aufgabe-2}

a)
\begin{quote}
1: main groß statt klein. Geschweifte Klammern statt Klammern im Funktionskopf.

2: Klammer statt geschweifte Klammer.

3: return groß statt klein. Klammern um 0. Semicolon nach Statement fehlt.

4: Runde Klammer statt geschweifte Klammer.

\begin{lstlisting}
int main(void)
{
  return 0;
}
\end{lstlisting}
\end{quote}

b)
\begin{quote}
1: Doppeltes Doppel-Kreuz.

3: Falscher Rückgabewert in main Funktion.

5: Runde Klammer nach printf fehlt. String nicht in Doppel-Hochkommata.


\begin{lstlisting}
#include <stdio.h>

int main(void)
{
  printf("Hallo");
}
\end{lstlisting}
\end{quote}
c)
\begin{quote}
0: stdio.h nicht inkludiert.

1: Runde Klammern nach main fehlen.

4: Falscher Funktionsname print. Falscher Format-Spezifizierer.


\begin{lstlisting}
#include <stdio.h>

int main(void)
{
  double d = 5.50;
  printf("%f",d);
  return 0;
}
\end{lstlisting}
\end{quote}
d)
\begin{quote}
1: Header falsch inkludiert. System Header immer mit < > inkludieren.

6: Ausgabe eines char ohne Format-Spezifizierer.

8: Geschweifte Klammer fehlt.


\begin{lstlisting}
#include <stdio.h>

int main(void)
{
  char d = `a`;
  printf("%c", d);
  return 0;
}
\end{lstlisting}
\end{quote}

\section{Aufgabe 4} 
 \label{sec:aufgabe-4}

a)
\begin{quote} 
\begin{lstlisting} 
int arithmetic_mean(int num1, int num2);
\end{lstlisting}
\end{quote}

\end{document}